{\color{red}JUNK AHEAD, just to show how to use latex commands}

The \Gls{latex} typesetting markup language is specially suitable for
documents that include \gls{maths}. \Glspl{formula} are rendered
properly an easily once one gets used to the commands.

Given a set of numbers, there are elementary methods to compute its
\acrlong{gcd}, which is abbreviated \acrshort{gcd}. This process is
similar to that used for the \acrfull{lcm}.

First use: \gls{API}\\
Subsequent: \gls{API}

Cite like this~\cite{miniboone_collaboration_significant_2018}, the
wiggle is to put a non-breakable space.

%%% Local Variables:
%%% mode: latex
%%% ispell-local-dictionary: "british"
%%% TeX-master: "../phd_thesis"
%%% End:
