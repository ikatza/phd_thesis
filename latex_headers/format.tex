% Purpose: set general formatting of the whole book

% TODO: make sure these numbers make sense for printing like a book
\geometry{
  a4paper,
  twoside=true,
  outer=2.9cm,
  inner=3.3cm,
  top=2.5cm,
  bottom=3.4cm,
  headsep=2.9mm,
  headheight=5.4mm,
}

\errorcontextlines=3

% For draft: put numbers
% \linenumbers

% Line spacing
\onehalfspacing
% \singlespacing


\pagestyle{fancy}
\setlength\voffset{1cm}
\fancyheadoffset[LE,RO]{12pt}
%\setlength\headsep{25pt} %TODO: what are you?

% Dutch style of paragraph formatting, i.e. no indents.
\setlength{\parskip}{1.3ex plus 0.2ex minus 0.2ex}
\setlength{\parindent}{0pt}

%%%%%%%%%%%%%%%%%%%%%%%%%%%%%%%%%%%%%%%%%%%%%%%%%%%%%%%
% To make sure it doesn't put headings on empty pages at the end of
% chapters that ended on odd page.
\makeatletter
\def\cleardoublepage{\clearpage
\if@twoside
  \ifodd\c@page\else
    \hbox{}
    \thispagestyle{plain} % set to empty so that the no page number
    \newpage
    \if@twocolumn\hbox{}\newpage\fi
  \fi
\fi}
\makeatletter
%%%%%%%%%%%%%%%%%%%%%%%%%%%%%%%%%%%%%%%%%%%%%%%%%%%%%%%%%
% Some other fancyheaders option.
% Foot notes on the exterior.
% On odd pages the chapter's tittle is display on the right and italics.
\pagestyle{fancy}
\fancyhf{}
\renewcommand{\chaptermark}[1]{\markboth{ \emph{#1}}{}}
\fancyhead[LO]{}
\fancyhead[RE]{\leftmark}
\fancyfoot[LE,RO]{\thepage}
\interfootnotelinepenalty=10000 %avoid breaking footnotes into two pages
%%%%%%%%%%%%%%%%%%%%%%%%%%%%%%%%%%%%%%%%%%%%%%%%%%%%%%%%%


% Modify title format of chapters
%%%%%%%%%%%%%%%%%%%%%%%%%%%%%%%%%%%%%%%%%%%%%%%%%%%%%%%%%
\renewcommand{\thechapter}{\Roman{chapter}}
\titleformat{\chapter}[display]
{\bfseries\Huge}
{\filleft\MakeUppercase{\chaptertitlename} \Huge\thechapter}
{4ex}
{\titlerule
  \vspace{2ex}%
  \filright}
[\vspace{2ex}%
\titlerule]
%%%%%%%%%%%%%%%%%%%%%%%%%%%%%%%%%%%%%%%%%%%%%%

%%%%%%%% Modifica formato de los titulos de los capitulos %%%%%%%
% Another Chapter Tittle Format Option
%%%%%%%%%%%%%%%%%%%%%%%%%%%%%%%%%%%%%%%%%%%%%%
% \titleformat{\chapter}[display] % cambiamos el formato de los capítulos
% {\bfseries\Huge} % por defecto se usarán caracteres de tamaño \Huge en negrita
% {% contenido de la etiqueta
% \titlerule[2pt] % línea horizontal
% \vspace{1pt}
% \filleft % texto alineado a la derecha
% \Large\chaptertitlename\ % "Capítulo" o "Apéndice" en tamaño \Large en lugar de \Huge
% \Large\thechapter} % número de capítulo en tamaño \Large
% {0mm} % espacio mínimo entre etiqueta y cuerpo
% {\filleft} % texto del cuerpo alineado a la derecha
% [\vspace{1pt} \noindent\rule{\textwidth}{2pt}]%\bigrule] % después del cuerpo, dejar espacio vertical y trazar línea horizontal gruesa
%%%%%%%%%%%%%%%%%%%%%%%%%%%%%%%%%%%%%%%%%%%%%%


\hypersetup{% no link boxes on TOC
  pdfauthor={\ppdfauthor},%
  pdfencoding=auto,%
  pdftitle={\ttitle},%
  unicode=true,%
  % pdftex,%
  psdextra,
  linktocpage,%make page number, not text, be link
  colorlinks,
  citecolor=dark-blue,%
  filecolor=black,%
  linkcolor=dark-green,%
  urlcolor=NavyBlue
  % menucolor
  % runcolor
  % anchorcolor
}

\captionsetup[figure]{labelfont=bf}
\captionsetup[table]{labelfont=bf}
% \captionsetup[lstlistlisting]{margin=0.5cm,font=footnotesize,labelfont=bf}
% \captionsetup[algorithmcf]{margin=0.5cm,font=footnotesize,labelfont=bf}
\captionsetup[subfigure]{labelfont=bf}
\captionsetup[subtable]{labelfont=bf}

\sidecaptionvpos{figure}{c} % t instead of c for top alignment


% \setcounter{secnumdepth}{2} %eg 1.1.1 in subsections
\setcounter{tocdepth}{2}
% 2: subsection
% 3: subsubsection
% 4:paragraph


% toc, tof, tol formatting
\usepackage[titles]{tocloft}
\tocloftpagestyle{fancy}
\setlength{\cftchapnumwidth}{3em}
\cftsetindents{section}{1em}{4em}
\cftsetindents{subsection}{1.5em}{3.5em}
\cftsetindents{figure}{1em}{4em}
\cftsetindents{table}{1em}{4em}

\pagenumbering{Roman} % roman page numbering i, ii, iii, ...

\AtBeginDocument{%
  \crefformat{section}{\S#2#1#3} % see manual of cleveref, section 8.2.1
  \crefformat{subsection}{\S#2#1#3}
  \crefformat{subsubsection}{\S#2#1#3}
  \crefmultiformat{section}{\S\S#2#1#3}{ and~#2#1#3}{, #2#1#3}{, and~#2#1#3}
}

\renewcommand{\thefootnote}{\roman{footnote}}
\DeclareNewFootnote{default}[roman]


% Another set of footnotes, use \footnoteB{}
% package footmisc is needed
\DefineFNsymbols*{atoms}{\atomA \atomB \atomC \atomD}
\DeclareNewFootnote{B}
\renewcommand{\thefootnoteB}{\fnsymbol{footnoteB}}
\MakeSortedPerPage{footnoteB}
\setfnsymbol{atoms}


% \defaultfontfeatures{ Scale=MatchLowercase, Ligatures = TeX }
% \setmainfont{XITS}[Scale = 1.0]
% \setmathfont{XITS Math}
% \setmathfont[range={scr,bfscr}, StylisticSet=1]{XITS Math}

\setmainfont{Latin Modern Roman}
\setmathfont{Latin Modern Math}
\setmathfont[range={scr,bfscr}, StylisticSet=1]{XITS Math}
% \setmathfont[range={}]{Latin Modern Math} % perhaps necessary if buggy?

% \setmathfont{STIX Two Math}
% \setmathfont[range={scr,bfscr}, Scale=MatchUppercase, StylisticSet=1]{STIX Two Math}

% \unimathsetup{math-style=TeX}


%%% Local Variables:
%%% mode: latex
%%% ispell-local-dictionary: "british"
%%% TeX-master: "../phd_thesis"
%%% End:
