% Purpose: define utilities or repeating phrases or strings

\makeatletter
\newcommand\listofillustrations{%
  % \chapter*{List of Figures, Tables, and Code}%
  \chapter*{List of Figures, and Tables}%
  % \phantomsection
  % \addcontentsline{toc}{chapter}{List of Figures, Tables, and Code}%
  \addcontentsline{toc}{chapter}{List of Figures, and Tables}%
  \section*{Figures}%
  \phantomsection
  \addcontentsline{toc}{section}{\protect\numberline{}Figures}%
  \@starttoc{lof}%
  \bigskip
  \section*{Tables}%
  \phantomsection
  \addcontentsline{toc}{section}{\protect\numberline{}Tables}%
  \@starttoc{lot}%
  \bigskip
  % \section*{Code}%
  % \phantomsection
  % \addcontentsline{toc}{section}{\protect\numberline{}Code}%
  % \@starttoc{loa}%
  % \bigskip
}
\makeatother

\newcommand{\cmark}{\ding{51}}%
\newcommand{\xmark}{\ding{55}}%
\newcommand{\atom}{\ding{105}}%
\newcommand{\atomA}{\ding{106}}%
\newcommand{\atomB}{\ding{90}}%
\newcommand{\atomC}{\ding{92}}%
\newcommand{\atomD}{\ding{103}}%
% \newcommand{\done}{\rlap{$\square$}{\raisebox{2pt}{\large\hspace{1pt}\cmark}}%
%   \hspace{-2.5pt}}
%\newcommand{\wontfix}{\rlap{$\square$}{\large\hspace{1pt}\xmark}}
\newcommand{\done}{{\color{green}{\Large \cmark}}}
\newcommand{\broken}{{\color{red}{\Large \xmark}}}

\newcommand{\bs}[1]{\symbf{#1}}

\newcommand{\bra}[1]{\left\langle #1\right|}
\newcommand{\ket}[1]{\left|#1\right\rangle}
\newcommand{\ketB}[1]{\big|#1\big\rangle}
\newcommand{\braket}[2]{\left\langle #1\middle|#2\right\rangle}
\newcommand{\braketB}[2]{\big\langle #1 \big|#2\big\rangle}

% units
\newcommand{\km}{~\mathrm{km}}
\newcommand{\m}{~\mathrm{m}}
\newcommand{\cm}{~\mathrm{cm}}
\newcommand{\mm}{~\mathrm{mm}}
\newcommand{\um}{~\mathrm{µm}}
\newcommand{\nm}{~\mathrm{nm}}
\newcommand{\s}{~\mathrm{s}}
\newcommand{\ms}{~\mathrm{ms}}
\newcommand{\us}{~\mathrm{µs}}
\newcommand{\ns}{~\mathrm{ns}}
\newcommand{\GeV}{~\mathrm{GeV}}
\newcommand{\MeV}{~\mathrm{MeV}}
\newcommand{\eV}{~\mathrm{eV}}
\newcommand{\eVV}{\eV^2}

\newcommand{\el}{\mathrm{e}}
\newcommand{\elec}{\mathrm{e^-}}
\newcommand{\posi}{\mathrm{e^+}}
\newcommand{\pr}{\mathrm{p}}
\newcommand{\prot}{\mathrm{p^+}}
\newcommand{\neu}{\mathrm{n}}
\newcommand{\neut}{\mathrm{n^0}}
\newcommand{\pinot}{\pi^0}
\newcommand{\pip}{\pi^+}
\newcommand{\pim}{\pi^-}
\newcommand{\pipm}{\pi^{\pm}}
\newcommand{\phot}{\gamma}

\newcommand{\nubar}{\bar{\nu}}
\newcommand{\nue}{\nu_{\el}}
\newcommand{\numu}{\nu_{\mu}}
\newcommand{\nutau}{\nu_{\tau}}
\newcommand{\nuebar}{\bar{\nu}_{\el}}
\newcommand{\numubar}{\bar{\nu}_{\mu}}
\newcommand{\numutonue}{\numu \rightarrow \nue}
\newcommand{\numubartonuebar}{\numubar \rightarrow \nuebar}

\newcommand{\Ar}{\mathrm{Ar}}

\newcommand{\Enu}{E_{\nu}}
\newcommand{\EnuQE}{E_{\nu}^{QE}}

\newcommand{\dk}{\mathcal{D}}
\newcommand{\dnu}{\mathcal{N_{\!\dk}}}
\newcommand{\nudk}{\nu_{\dk}}
\newcommand{\nudkbar}{\bar{\nu}_{\dk}}
\newcommand{\zdk}{Z_{\dk}}

\newcommand{\dmm}{\Delta m^2}
\newcommand{\stt}{\sin^2{ 2 \theta}}
\newcommand{\eVVScale}{\sim1\eVV}
\newcommand{\thetaOneTwo}{\theta_{12}}
\newcommand{\deltaCP}{\delta^{\mathrm{CP}}}

\newcommand{\GENIE}{\texttt{GENIE}}
\newcommand{\LArSoft}{\texttt{LArSoft}}
\newcommand{\SBNDCode}{\texttt{SBNDCode}}
\newcommand{\ROOT}{\texttt{ROOT}}


% \newcommand{\pspace}{\left(\sin^2{ 2 \theta}, \Delta m^2\right)}
% \newcommand{\pplane}{\sin^2{ 2 \theta}\mbox{--}\Delta m^2}
% \newcommand{\data}{\bs{D}}
% \newcommand{\bgr}{\bs{B}}
% \newcommand{\signal}{\bs{S}}
% \newcommand{\pred}{\bs{Q}}
% \newcommand{\like}{\mathscr{L}}
% \newcommand{\true}{\Delta \mathscr{L_{\mathbf{V}}}}
% \newcommand{\true}{\Delta \mathscr{L}}
% \newcommand{\fake}{\Delta \like_{\mathbf{f}\ i}^{\{\bs{p}_0\}}}
% \newcommand{\eff}{\chi^2_{\mathbf{Ef}}}%no me gusta esta notacion!!!!
% \newcommand{\NF}{n_{\mathbf{f}}}
% \newcommand{\critic}{\Delta\like^{\{\bs{p}_0\}}_c}


\newcommand\scalemath[2]{\scalebox{#1}{\mbox{\ensuremath{\displaystyle #2}}}}

\DeclareMathAlphabet{\mathcalligra}{T1}{calligra}{m}{n}
\DeclareFontShape{T1}{calligra}{m}{n}{<->s*[2.2]callig15}{}
\DeclareRobustCommand{\scriptr}{%
  \mspace{-2mu}%
  \text{\usefont{T1}{calligra}{m}{n}r\/}%
  \mspace{2mu}%
}


\definecolor{dark-red}{rgb}{0.4,0.15,0.15}
\definecolor{dark-blue}{rgb}{0.15,0.15,0.4}
\definecolor{medium-blue}{rgb}{0,0,0.5}
\definecolor{dark-green}{RGB}{50,71,13}
\definecolor{light-gray}{gray}{0.95}

\newcommand{\code}[1]{\colorbox{light-gray}{\texttt{#1}}}

\newcommand{\dummytext}{{\color{yellow}\blindtext}}
\newcommand{\domeetext}{{\color{green}\blindtext}}

\newcommand{\wittyremark}[1]{%
  % \renewcommand{\thefootnoteB}{\ding{106}}%so that is always the same
  \footnoteB{#1}%
}


% to be able to compile with or without hyperref without an error
\providecommand\phantomsection{}


% To have left, centre and right columns with a specific width in tabular
\newcolumntype{L}[1]{>{\raggedright\let\newline\\\arraybackslash\hspace{0pt}}m{#1}}
\newcolumntype{C}[1]{>{\centering\let\newline\\\arraybackslash\hspace{0pt}}m{#1}}
\newcolumntype{R}[1]{>{\raggedleft\let\newline\\\arraybackslash\hspace{0pt}}m{#1}}


%%% Local Variables:
%%% mode: latex
%%% TeX-master: "../phd_thesis"
%%% End:
